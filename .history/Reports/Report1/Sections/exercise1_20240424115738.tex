\subsection{Astrophysical context} \label{astrophysical_context}
Black holes (BHs) are astrophysical objects that create a gravitational field so intense that it prevents anything from escaping them, even light. For this reason, they are very difficult to study, especially those that do not have an accretion disk. A significant breakthrough occurred when the first gravitational-wave signal was observed, in September 2015 by the LIGO interferometers. These signals encode a lot of precious information about their sources, enabling us to study binaries of stellar-mass BHs as never before. For instance, estimates of BH masses and, as we will discuss, of spin-related quantities can be extracted from these waveforms. It should be pointed out that only the most compact BH binaries can be studied in this way, because the gravitational-wave emission timescale is proportional to $r^4$, where $r$ is the orbital separation of the binary, and thus a small enough $r$ is a necessary condition to have a merging event in a time shorter than the life of the universe.

\vspace{4mm}  


